Designing an architectural or urban plan is a complex problem in which a good solution is one that balances aesthetics, functionality, utilization, and safety of a structure. This makes the problem inherently multiobjective and the solution space combinatorial.  Incorporating the dynamics of large groups of people, or crowds, further complicate the solution space and is prohibitively expensive to do with real people.  Thus layout designs are commonly tested using synthetic crowds that model realistic behaviours under different conditions.  In particular, a layout's performance is most critical during dangerous situations such as evacuations, so these scenarios are used to stress test environments.

Popular architectural design tools such as Autodesk Revit and Rhino3D do not account for evacuation planning or crowd-aware stress testing. There are tools used to analyze a design with simulated crowds, but none of those tool have an integrated environment for both architectural design and crowd simulation. Together these tools are quite complex an require years of training and architectural and safety knowledge.

Fully automated computational solutions for crowd-aware architectural design may miss solutions which are technically sound, crowd-aware, and aesthetically pleasing.  These are highly dependant on the quality of the objective functions for each of these requirements. Recent work in this field has sought for user-in-the-loop optimization processes which make up for these shortcoming and provide the user more control over solution directions.

Our solution is to gamify the process of optimal layout design for crowd-aware architectures, or environments. This game provides the player with feedback in terms of crowd simulation and heatmaps of evacuation dynamics, while affording them editing power within the constraints imposed by the architect or designer. The approach of gamifying complicated problems affords many benefits to both the process and the solution. The most immediate is that the game provides a fun and interactive platform for architectural and urban planning. 

By providing a means to implement the design process as gamified levels with built in architectural constraints, a planner, environment designer, or architect can convert their design problem into a playable game.  This reimagines the architect or planner seeking a design solution as a game maker.

Furthermore, providing the game via the internet, or large scale network of collaborators, design solutions may be crowd sourced.  A user with any level of knowledge in a related field can produce an optimal design plan. This can also be used as a exciting learning platform in the field of architectural education. 

Providing the results of other players designs and their fitness affords a further level of colaboration in the form of competitive co-design.  The collective knowledge of the crowd, as in collaborators, and iterative improvement of co-design will help the layout to converge towards optimality. 