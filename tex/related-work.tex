There is an established and growing interest in the use of architectural optimization to explore design spaces and provide optimal solutions with respect to problem criteria~\cite{block2014advances,pottmann2014architectural}. While there are popular CAD solution tools, such as Autodesk Revit and Rhino3D, these tools do not account for crowd behaviour and require significant training and background knowledge. Architectural optimization solutions generate new layouts or topology for structures with respect to objective and/or subjective criteria, while providing a trade off between automation and author precision.  Data driven approaches~\cite{Merrell:2010:CRB:1882261.1866203} learn layout configurations from a database of prior architecture design constrained to a particular design space (e.g., residential homes). In another approach Design objectives can be modelled as forces applied to a physical features to generate layout designs automatically~\cite{arvin2002modeling}.

\red{Some editing has been done here to remove generic references (like theses).  THis section should be about either crowd awre optimization OR crowd simulation, and NOT both}~There are lot of existing research on crowd simulation. The Social Force model allows for highly representative models of human behaviour in simple situations that elicit reflexive reactions~\cite{PhysRevE.51.4282}. The most closely related to our work is the optimization of egress environments using crowds~\cite{johansson2007pedestrian,jiang2014obstacle}. This work does not incorporate the user-in-the-loop processes and is presented as analysis of the affects of egress obstacles and shapes on crowd evacuation. Since subjective criteria are difficult or impossible to quantify, many tools select an optimization scheme to meet objective criteria and then take a human-in-the-loop interactive approach to the subjective. The tools combine the aforementioned derivations for fitness with user guided optimization processes ~\cite{shi2013performance,turrin2011design,Michalek02interactivedesign}. 

\red{THis and the previous paragraph are not grouped properly, we have crowd simulation from generic overview (badler book) to specific (sf + ppr +etc), crowd aware optimization, and then parameter fitting blended over two paragraphs - there should be three separately with clear focus}~The maturity of research in simulating crowd dynamics~\cite{badlerBook,DBLP:books/daglib/0030710} has resulted in a wide variety of approaches including social forces~\cite{helbing2000simulating} and predictive models~\cite{ORCA,ppr}. There has been a growing recent trend to use statistical analysis in the evaluation and analysis of dense crowd simulations.  The work in~\cite{guy2012statistical,Pettre:2009:EMS:1599470.1599495} measures the ability of a steering algorithm to emulate the behavior of a real crowd data set by measuring its divergence from ground truth.  Crowd optimization techniques~\cite{paramFitting,sca.20141129} automatically fit the parameters of a crowd to meet different criteria, or even match real crowds. 

\red{Cite LoS paper here somewhere in terms of the combinatorial decision space of the non-convex optimization problem that is architectural optimization.  Also, cite previous CAVW paper on environment optimization.}

We leverage these wealth of research in crowd simulation to present a computational design tool for environments that uses features extracted from an underlying crowd simulator as criteria for architectural optimization.

There is a plethora of research on crowd simulation and crowd evacuation using serious games. Most closest to our approach is  ~\cite{ribeiro2013towards} where serious games are used to training, planning and evaluating emergency plans but the research more focuses on crowd simulation component than architectural planning. Our Game focuses on optimizing the architectural design keeping in consideration the emergency situation using best suited simulated crowd. So that we can prevent a disaster at an earlier stage. Our game also considers the  research paper~\cite{berseth2014characterizing} for characterizing and optimizing game level difficulty for our architectural planning game~\red{"considers" how?}.