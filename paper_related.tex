\noindent \textbf{Architectural Optimization.} There is an established and growing interest in the use of architectural optimization to explore design spaces and provide optimal solutions with respect to problem criteria~\cite{block2014advances,pottmann2014architectural}. 
Architectural optimization solutions generate new layouts or topology for structures with respect to objective and/or subjective criteria, while providing a trade off between automation and author precision.  Data driven approaches~\cite{Merrell:2010:CRB:1882261.1866203} learn layout configurations from a database of prior architecture design constrained to a particular design space (e.g., residential homes).  Another approach is to model objectives as relationships between features. Design objectives can be modeled as forces applied to a physical features to generate layout designs automatically ~\cite{arvin2002modeling}.  Hierarchical and spatial relationships of furniture objects can be modeled to produce realistic human designer-like configurations with respect to their computed visibility and accessibility ~\cite{craigyu2011furniture}.  

The modeling of physical phenomenon is most related to our approach to environment fitness.  These methods may incorporate simulation of sunlight~\cite{yi2014performance}, materials, subsequent energy savings~\cite{caldas2002design}, or even acoustics~\cite{bassuet2014computational}.  The most closely related to our work is the optimization of egress environments using crowds~\cite{johansson2007pedestrian, jiang2014obstacle}.  This work does not incorporate the user-in-the-loop processes and is presented as analyses of the affects of egress obstacles and shapes on crowd evacuation.

Since subjective criteria are difficult or impossible to quantify, many tools select an optimization scheme to meet objective criteria then take a human-in-the-loop interactive approach to the subjective.  The tools combine the aforementioned derivations for fitness with user guided optimization processes ~\cite{shi2013performance,turrin2011design,Michalek02interactivedesign}. 

\noindent \textbf{Crowd Simulation, Analysis, and Optimization.} The maturity of research in simulating crowd dynamics~\cite{badlerBook,DBLP:books/daglib/0030710} has resulted in a wide variety of approaches including social forces~\cite{helbing2000simulating} and predictive models~\cite{ORCA,ppr}. There has been a growing recent trend to use statistical analysis in the evaluation and analysis of dense crowd simulations.  The work in~\cite{guy2012statistical,Pettre:2009:EMS:1599470.1599495} measures the ability of a steering algorithm to emulate the behavior of a real crowd data-set by measuring its divergence from ground truth.  Crowd optimization techniques~\cite{paramFitting,sca.20141129} automatically fit the parameters of a crowd to meet different criteria, or even match real crowds. 

\noindent \textbf{Our Work.} We leverage the wealth of research in crowd simulation to present a computational design tool for environments that uses features extracted from an underlying crowd simulator as criteria for architectural optimization.  
